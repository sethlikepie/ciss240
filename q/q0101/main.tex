%-*-latex-*-
%-*-latex-*-
\newcommand\COURSE{ciss240}
\newcommand\ASSESSMENT{q0304}
\newcommand\ASSESSMENTTYPE{Quiz}
\newcommand\POINTS{\textwhite{xxx/xxx}}

\input{myquizpreamble}
\input{yliow}
\input{\COURSE}
%-*-latex-*-

%-*-latex-*-

%-*-latex-*-
\renewcommand\TITLE{\ASSESSMENTTYPE \ \ASSESSMENT}

\renewcommand\EMAIL{}
\renewcommand\AUTHOR{sithurman1@cougars.ccis.edu}

\textwidth=6in
\begin{document}
\topmatterthree

This is a closed-book, no compiler, 5 minute quiz.

This is a closed book quiz.
You have 5 minutes (that is from beginning to the end including
the submission of your work.)

If you are using Microsoft Visual Studio, you must clear all the
auto-generated code and write the program on your own.

%------------------------------------------------------------------------------
\nextq
The goal is to write a \cpp\ program.

The FIRST FEW LINES of your \cpp\ source file must look like this:
\begin{console}
// Name: John Doe
// File: main.cpp

#include <iostream>
\end{console}
with \lq\lq John Doe''replaced by your name (of course).

Write a \cpp\ program that produces the following output in the console
window (when you execute the program):
\begin{console}
hello master ...
\end{console}
Your output must following the above \textit{exactly} as given.
For instance,
your program is \textit{wrong} if the output is
\begin{console}
Hello master ...
\end{console}
or
\begin{console}
hello master.
\end{console}
or
\begin{console}
hello    master ...
\end{console}
or any other variation.

Your coding style must following the coding style as used in class.
That includes the spacing, the blank lines, etc.

After you have tested your code, open \verb!main.tex! (using emacs).
Look for \verb!ANSWER! (in emacs search is \verb!C-s!).
Copy-and-paste your code between \verb!\begin{answercode}! and
\verb!\end{answercode}!.
Save \verb!main.tex! (in emacs do \verb!C-x C-s!).
In your bash shell, execute \verb!make! and view the pdf.
Submit using the \verb!alex! program.

\ANSWER
\begin{answercode}
// Name: Seth Thurman
// File: main.cpp

#include <iostream>

int main()
{
  std::cout << "hello master ...\n";

  return 0;
}
\end{answercode}

\textsc{Grading.}
\begin{enumerate}
\item If your program was not received in time: 0/2
\item If your program contains error(s) and does not run: 0/2
\item If your program is error-free, does run, but produces no output: 0/2
\item If your program is error-free, does run, produces an output, but the output does not match the given output: 0/2
\item If your program is error-free, does run, and produces an output that matches
the given output: 2/2
\item After the points from above,
point(s) will be taken off for incorrect coding style.
\end{enumerate}

%------------------------------------------------------------------------------
%-*-latex-*-
\newpage
\input{instructions}
\end{document}

