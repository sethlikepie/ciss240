%-*-latex-*-
%-*-latex-*-
\newcommand\COURSE{ciss240}
\newcommand\ASSESSMENT{q0304}
\newcommand\ASSESSMENTTYPE{Quiz}
\newcommand\POINTS{\textwhite{xxx/xxx}}

\input{myquizpreamble}
\input{yliow}
\input{\COURSE}
%-*-latex-*-

%-*-latex-*-

%-*-latex-*-
\renewcommand\TITLE{\ASSESSMENTTYPE \ \ASSESSMENT}

\renewcommand\EMAIL{}
\renewcommand\AUTHOR{sithurman1@cougars.ccis.edu}

\textwidth=6in
\begin{document}
\topmatterthree

This is a closed-book, no compiler, 5 minute quiz.

%------------------------------------------------------------------------------
\nextq
What is the type of the value of the following expression?
\begin{console}[fontsize=\small]
1 + 2 * 3.1 - 4 + 3
\end{console}
(Write \verb!int! or \verb!char! or \verb!C-string! or \verb!float! or
\verb!double!.)
\\
\ANSWER
\begin{answercode}
double
\end{answercode}

%------------------------------------------------------------------------------
\nextq
Joe Cantcode wrote the following code fragment:
\begin{console}[commandchars=\@\{\},fontsize=\small]
int x, y;
std::cin >> x >> y;
std::cout << x / y << std::endl;
\end{console}
When he entered \verb!2! for \verb!x! and \verb!8! for \verb!y!,
the instructor wanted
the output to be 0.25.
But it's \textit{not}.
What should be printed in the print statement:
\begin{console}[commandchars=\@\{\},fontsize=\small]
int x, y;
std::cin >> x >> y;
std::cout << @underline{@hspace{2in}} << std::endl;
\end{console}
\ANSWER
\begin{answercode}
double(x) / y
\end{answercode}

%------------------------------------------------------------------------------
\nextq
What integer value in the following
C\texttt{++}
expression is the \textit{first} to be type promoted to a \texttt{double}?
\begin{console}[commandchars=\\\{\},fontsize=\small]
4 + 2 - 13 / 4 * 5.1 + 2
\end{console}
\ANSWER
\begin{answercode}
3
\end{answercode}

%------------------------------------------------------------------------------
\nextq
What is the final value of \verb!s! for the following code fragment?
\begin{Verbatim}[fontsize=\footnotesize,frame=single]
double s = 0.0;
int i = 1;
int j = 10;

s = s + i / double(j);
i = i + 1;
j = j * 10;

s = s + i / double(j);
i = i + 1;
j = j * 10;

s = s + i / double(j);
i = i + 1;
j = j * 10;

s = s + i / double(j);
i = i + 1;
j = j * 10;

s = s + i / double(j);
i = i + 1;
j = j * 10;
\end{Verbatim}
\ANSWER
\begin{answercode}
0.12345
\end{answercode}

%------------------------------------------------------------------------------
%-*-latex-*-
\newpage
\input{instructions}
\end{document}

