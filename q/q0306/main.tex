%-*-latex-*-
%-*-latex-*-
\newcommand\COURSE{ciss240}
\newcommand\ASSESSMENT{q0202}
\newcommand\ASSESSMENTTYPE{Quiz}
\newcommand\POINTS{\textwhite{xxx/xxx}}

\input{myquizpreamble}
\input{yliow}
\input{\COURSE}
\input{thispackages}
\input{thismacros}
\input{thistitle}
\renewcommand\EMAIL{}
\renewcommand\AUTHOR{sithurman1@cougars.ccis.edu}

\textwidth=6in
\begin{document}
\topmatterthree

This is a closed-book, no compiler, 5 minute quiz.

%------------------------------------------------------------------------------
\nextq
The following code fragment has a repeating chunk of code that repeats
3 times:
\begin{Verbatim}[fontsize=\footnotesize,frame=single,commandchars=\\\{\}]
int i = 1, j = 1;

j = j * i;
i = i + 1;

j = j * i;
i = i + 1;

j = j * i;
i = i + 1;
\end{Verbatim}
If the goal is to compute $1 \times 2 \times 3 \times \cdots \times 8$,
how many times does the repeating chunk of code appear?
\\
\ANSWER
\begin{answercode}
8
\end{answercode}

%------------------------------------------------------------------------------
\nextq
What is the final value of \verb!i! at the end of this code fragment?
\begin{Verbatim}[fontsize=\footnotesize,frame=single]
int i = 0, j = 1;

i = i * 10 + j;
j = j + 1;

i = i * 10 + j;
j = j + 1;

i = i * 10 + j;
j = j + 1;

i = i * 10 + j;
j = j + 1;

i = i * 10 + j;
j = j + 1;

i = i * 10 + j;
j = j + 1;
\end{Verbatim}
\ANSWER
\begin{answercode}
123456
\end{answercode}

%------------------------------------------------------------------------------
\nextq
What is the final value of \verb!k! at the end of this code fragment?
\begin{Verbatim}[fontsize=\footnotesize,frame=single]
int i = 135692468, j = 1000000, k;

k = i / j % 10;
j = j / 10;

k = i / j % 10;
j = j / 10;

k = i / j % 10;
j = j / 10;

k = i / j % 10;
j = j / 10;

k = i / j % 10;
j = j / 10;

k = i / j % 10;
j = j / 10;
\end{Verbatim}
\ANSWER
\begin{answercode}
6
\end{answercode}

%------------------------------------------------------------------------------
\nextq
What is the final value of \verb!k! at the end of this code fragment?
\begin{Verbatim}[fontsize=\footnotesize,frame=single]
int i = 135792468, j = 1, k = 0;

k = k * 10 + i / j % 10;
j = j * 10;

k = k * 10 + i / j % 10;
j = j * 10;

k = k * 10 + i / j % 10;
j = j * 10;

k = k * 10 + i / j % 10;
j = j * 10;

k = k * 10 + i / j % 10;
j = j * 10;

k = k * 10 + i / j % 10;
j = j * 10;
\end{Verbatim}
\ANSWER
\begin{answercode}
864297
\end{answercode}

%------------------------------------------------------------------------------
\nextq
What is the final value of \verb!k! at the end of this code fragment?
\begin{Verbatim}[fontsize=\footnotesize,frame=single]
int i = 135792468, j = 1, k = 0;

k = k + i / j % 10;
j = j * 10;

k = k + i / j % 10;
j = j * 10;

k = k + i / j % 10;
j = j * 10;

k = k + i / j % 10;
j = j * 10;

k = k + i / j % 10;
j = j * 10;

k = k + i / j % 10;
j = j * 10;
\end{Verbatim}
\ANSWER
\begin{answercode}
36
\end{answercode}

%------------------------------------------------------------------------------
\nextq
Complete the following code segment so that
\verb!i! and \verb!j!
swap their values (you can declare an extra variable):
\begin{console}[fontsize=\footnotesize]
int i = 0, j = 1;

// ENTER CODE BELOW
t = i;
i = j;
j = t;

std::cout << i << ' ' << j;
\end{console}
i.e. the output of the above code segment must be
\begin{console}[fontsize=\footnotesize]
1 0
\end{console}
If I change the initial values for \verb!i!, \verb!j! to \verb!1!, \verb!2!,
the output is \verb!2 1!.
\\
\ANSWER
\begin{answercode}
t = i;
i = j;
j = t;
\end{answercode}

%------------------------------------------------------------------------------
\nextq
Complete the following code segment
\begin{console}[fontsize=\footnotesize]
int x0 = 0, x1 = 1, x2 = 2;
int t;

// ENTER CODE BELOW
t = x0;
x0 = x1;
x1 = x2;
x2 = t;

std::cout << x0 << ' ' << x1 << ' ' << x2 << '\n';
\end{console}
so that you get the following output
\begin{console}[fontsize=\footnotesize]
1 2 0
\end{console}
And if the initial values for
\verb!x0!,
\verb!x1!,
\verb!x2!,
are
\verb!3!,
\verb!4!,
\verb!5!,
the output is \verb!4 5 3!.
Do it in 4 statements.
\\
\ANSWER
\begin{answercode}
t = x0;
x0 = x1;
x1 = x2;
x2 = t;
\end{answercode}

%------------------------------------------------------------------------------
\input{thispostamble}
