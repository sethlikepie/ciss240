%-*-latex-*-
%-*-latex-*-
\newcommand\COURSE{ciss240}
\newcommand\ASSESSMENT{q0202}
\newcommand\ASSESSMENTTYPE{Quiz}
\newcommand\POINTS{\textwhite{xxx/xxx}}

\input{myquizpreamble}
\input{yliow}
\input{\COURSE}
\input{thispackages}
\input{thismacros}
\input{thistitle}
\renewcommand\EMAIL{}
\renewcommand\AUTHOR{sithurman1@cougars.ccis.edu}

\textwidth=6in
\begin{document}
\topmatterthree

This is a closed-book, no compiler, 5 minute quiz.

%------------------------------------------------------------------------------
\nextq
What is the output of following code fragment:
\begin{Verbatim}[frame=single,fontsize=\footnotesize]
std::cout << 135246 / 10000 << std::endl;
\end{Verbatim}
\ANSWER
\begin{answercode}
13
\end{answercode}

%------------------------------------------------------------------------------
\nextq
What is the output of following code fragment:
\begin{Verbatim}[frame=single,fontsize=\footnotesize]
std::cout << 135246 % 100 << std::endl;
\end{Verbatim}
\ANSWER
\begin{answercode}
46
\end{answercode}

%------------------------------------------------------------------------------
\nextq
What is the output of the following code fragment:
\begin{Verbatim}[frame=single,fontsize=\footnotesize]
std::cout << (1357246 / 10000 % 100)  << std::endl;
\end{Verbatim}
\ANSWER
\begin{answercode}
35
\end{answercode}

%------------------------------------------------------------------------------
\nextq
What is the integer printed by the following code fragment:
\begin{Verbatim}[frame=single,fontsize=\footnotesize]
std::cout << (1357246 % 10000 / 100)  << std::endl;
\end{Verbatim}
\ANSWER
\begin{answercode}
24
\end{answercode}

%------------------------------------------------------------------------------
\nextq
To print the 4th digit from the right of \verb!1357246! (which is \verb!7!),
I can execute this code fragment
\begin{Verbatim}[frame=single,fontsize=\footnotesize]
std::cout << (1357246 / x % 10) << std::endl;
\end{Verbatim}
where \verb!x! is a 10-power
(i.e., \verb!x! is \verb!1! or \verb!10! or \verb!100! or \verb!1000!
or \verb!10000!, etc.)
What is the value of \verb!x!?
\\
\ANSWER
\begin{answercode}
1000
\end{answercode}

%------------------------------------------------------------------------------
\newpage
\nextq
To check if 35 is a prime, I can execute the following
code fragment:
\begin{Verbatim}[frame=single,fontsize=\footnotesize]
std::cout << 35 % 2 << '\n';
std::cout << 35 % 3 << '\n';
std::cout << 35 % 4 << '\n';
std::cout << 35 % 5 << '\n';
std::cout << 35 % 6 << '\n';
std::cout << 35 % 7 << '\n';
std::cout << 35 % 8 << '\n';
std::cout << 35 % 9 << '\n';
std::cout << 35 % 10 << '\n';
std::cout << 35 % 11 << '\n';
std::cout << 35 % 12 << '\n';
std::cout << 35 % 13 << '\n';
std::cout << 35 % 14 << '\n';
std::cout << 35 % 15 << '\n';
std::cout << 35 % 16 << '\n';
std::cout << 35 % 17 << '\n';
std::cout << 35 % 18 << '\n';
std::cout << 35 % 19 << '\n';
std::cout << 35 % 20 << '\n';
std::cout << 35 % 21 << '\n';
std::cout << 35 % 22 << '\n';
std::cout << 35 % 23 << '\n';
std::cout << 35 % 24 << '\n';
std::cout << 35 % 25 << '\n';
std::cout << 35 % 26 << '\n';
std::cout << 35 % 27 << '\n';
std::cout << 35 % 28 << '\n';
std::cout << 35 % 29 << '\n';
std::cout << 35 % 30 << '\n';
std::cout << 35 % 31 << '\n';
std::cout << 35 % 32 << '\n';
std::cout << 35 % 33 << '\n';
std::cout << 35 % 34 << '\n';
\end{Verbatim}
But in fact I can stop earlier.
What is the smallest value of \verb!d! such that
I can stop at \verb!35 % d!?
\\
\ANSWER
\begin{answercode}
5
\end{answercode}

%------------------------------------------------------------------------------
\input{thispostamble}
